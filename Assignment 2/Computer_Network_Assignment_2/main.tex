%!TeX program = xelatex
\documentclass[12pt,hyperref,a4paper,UTF8]{ctexart}

\usepackage{homework}

%%-------------------------------正文开始---------------------------%%
\begin{document}

%%-----------------------封面--------------------%%
\cover

%%------------------摘要-------------%%
%\begin{abstract}
%
%在此填写摘要内容
%
%\end{abstract}

\thispagestyle{empty} % 首页不显示页码

%%--------------------------目录页------------------------%%
\newpage
\tableofcontents

%%------------------------正文页从这里开始-------------------%
\newpage

%%可选择这里也放一个标题
%\begin{center}
%    \title{ \Huge \textbf{{标题}}}
%\end{center}

% ==============
% =    规则    =
% ==============
\section{规则}
\begin{enumerate}[I]
    \item 在规定时间内不能提交作业者,零分;

    \item 没有推导过程,只列出答案者,零分;

    \item 格式混乱,无法阅读者,零分;

    \item 不懂就问,不会就学。任何经验都是后天积累的。

    \item 关于LaTeX的一切,国内外的网站都有详尽的教学视频。例:
    \begin{itemize}
        \item \href{https://www.bilibili.com/video/BV1Jy4y1p76e/?vd_source=2c0f6624843da61e86c7e8a2b75de875}{Bilibili}

        \item \href{https://www.youtube.com/watch?v=Jp0lPj2-DQA&list=PLHXZ9OQGMqxcWWkx2DMnQmj5os2X5ZR73}{YouTube}
    \end{itemize}
\end{enumerate}

\newpage

% ==============
% =    习题    =
% ==============
\section{习题}

\subsection{物理层的概念}
\textbf{问:}物理层要解决哪些问题?物理层的主要特点是什么?

\textbf{答:}物理层要解决的主要问题:

(1)物理层要尽可能地屏蔽掉物理设备和传输媒体,通信手段的不同,使数据链路层感觉不到这些差异,只考虑完成本层的协议和服务。

(2)给其服务用户(数据链路层)在一条物理的传输媒体上传送和接收比特流(一般为串行按顺序传输的比特流)的能力,为此,物理层应该解决物理连接的建立、维持和释放问题。

(3)在两个相邻系统之间唯一地标识数据电路

物理层的主要特点:

(1)由于在OSI之前,许多物理规程或协议已经制定出来了,而且在数据通信领域中,这些物理规程已被许多商品化的设备所采用,加之,物理层协议涉及的范围广泛,所以至今没有按OSI的抽象模型制定一套新的物理层协议,而是沿用已存在的物理规程,将物理层确定为描述与传输媒体接口的机械,电气,功能和规程特性。

(2)由于物理连接的方式很多,传输媒体的种类也很多,因此,具体的物理协议相当复杂。

\subsection{物理层的接口}
\textbf{问:}物理层的接口有哪几个方面的特性?各包含什么内容?

\textbf{答:}(1)机械特性:明接口所用的接线器的形状和尺寸、引线数目和排列、固定和锁定装置等等。

(2)电气特性:指明在接口电缆的各条线上出现的电压的范围。

(3)功能特性:指明某条线上出现的某一电平的电压表示何意。

(4)规程特性:说明对于不同功能的各种可能事件的出现顺序。

\subsection{码元与振幅调制}
\textbf{问:}假定某信道受奈氏准则限制的最高码元速率为$20,000$码元/秒。如果采用振幅调制,把码元的振幅划分为16个不同等级来传送,那么可以获得多高的数据率(bit/s)?

\textbf{答:}由于振幅调制将码元的振幅划分为 16 个不同等级,因此每个等级可以表示一个不同的状态。16个不同等级代表4个比特为一组,则原来的一个码元可以表示现在的四个码元。并根据奈氏准则,可以得出下列公式:
\begin{equation}\label{eq:2.3}
\begin{aligned}
     C = R \times \log_2 M = 20,000 \times \log_2 16 = 80,000\text{{bit/s}}
\end{aligned}
\end{equation}

\subsection{信号衰减}
\textbf{问:}假定有一种双绞线的衰减是$0.7$dB/km(在1kHz时),若容许有$20$dB的衰减,试问
\begin{itemize}
    \item 使用这种双绞线的链路的工作距离有多长?

    \item 如果要使这种双绞线的工作距离增大到$100$km,问应当使衰减降低到多少?
\end{itemize}


\textbf{(1)答:} 已知双绞线的衰减为 0.7dB/km,容许的最大衰减为 20dB。最大工作距离 L 可以通过以下公式计算
\begin{equation}\label{eq:2.4a}
\begin{aligned}
  L = \frac{\text{容许衰减}}{\text{衰减率}} = \frac{20\text{dB}}{0.7\text{dB/km}} = 28.57\text{{km}}
\end{aligned}
\end{equation}
\textbf{(2)答:} 已知容许的最大衰减为 20dB,工作距离提升至100km,衰减率decay应降低至:
\begin{equation}\label{eq:2.4b}
\begin{aligned}
   decay = \frac{\text{容许衰减}}{\text{工作距离}} = \frac{20dB}{100Km} = 0.2\text{{dB/km}}
\end{aligned}
\end{equation}


\subsection{信道复用}
\textbf{问:}为什么要使用信道复用技术?常用的信道复用技术有哪些?

\textbf{答:}信道复用技术可以将多个信号通过一个信道传输,从而提高信道的利用率。不同的信道复用技术可以满足不同的通信需求,可以提高传输速率、保证通信质量等。

常用的信道复用技术包括:

1.时分复用(TDM):将时间划分成若干个时隙,不同的信号在不同的时隙中传输,如电话系统中的数字化交换机。

2.频分复用(FDM):将频率划分成若干个频带,不同的信号在不同的频带中传输,如广播电视系统。

3.码分复用(CDM):采用不同的扩频码将不同的信号进行编码,使它们在同一频带上传输,如CDMA系统。

4.波分复用(WDM):利用不同的波长将不同的信号分离,使它们在同一光纤上传输,如光通信系统。

\subsection{CDMA}
\textbf{问:}共有$4$个站进行码分多址CDMA通信,四个站的码片序列如\autoref{tab:cdma}所示。
现收到这样的码片序列:$(-1, +1, -3, +1, -1, -3, +1, +1)$。
问
\begin{itemize}
    \item 哪个站发送数据了?
    
    \item 发送数据的站发送的是$1$还是$0$?
\end{itemize}

\newpage
\begin{table}[t!]
    \centering
    \begin{tabular}{cc}
       A: $(-1, -1, -1, +1, +1, -1, +1, +1)$  & B: $(-1, -1, +1, -1, +1, +1, +1, -1)$ \\
       C: $(-1, +1, -1, +1, +1, +1, -1, -1)$  & D: $(-1, +1, -1, -1, -1, -1, +1, -1)$
    \end{tabular}
    \caption{码片序列}
    \label{tab:cdma}
\end{table}

\textbf{答:}A和D站发送1,B发送0,C未发送数据
\begin{equation}\label{eq:2.6a}
\begin{aligned}
    A: \frac{(-1) \times (-1) + (-1) \times 1 + (-1) \times (-3) + 1 \times 1 +1 \times (-1) + (-1) \times (-3) + 1 \times 1 + 1 \times 1 }{8} = 1
\end{aligned}
\end{equation}

\begin{equation}\label{eq:2.6b}
\begin{aligned}
    B: \frac{(-1) \times (-1) + (-1) \times 1 + 1 \times (-3) + (-1) \times 1 +1 \times (-1) + 1 \times (-3) + 1 \times 1 + (-1) \times 1 }{8} = -1
\end{aligned}
\end{equation}

\begin{equation}\label{eq:2.6c}
\begin{aligned}
    C: \frac{(-1) \times (-1) + 1 \times 1 + (-1) \times (-3) + 1 \times 1 +1 \times (-1) + 1 \times (-3) + -(1) \times 1 + (-1) \times 1 }{8} = 0
\end{aligned}
\end{equation}

\begin{equation}\label{eq:2.6d}
\begin{aligned}
    D:\frac{(-1) \times (-1) + 1 \times 1 + (-1) \times (-3) + (-1) \times 1 + (-1) \times (-1) + (-1) \times (-3) + 1 \times 1 + (-1) \times 1}{8}\\ =1 
\end{aligned}
\end{equation}



\end{document}