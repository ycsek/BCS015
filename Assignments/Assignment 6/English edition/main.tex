% !TEX program = xelatex
\documentclass[12pt,hyperref,a4paper,UTF8]{ctexart}
\usepackage{CityUhomework}
\usepackage{booktabs}
\usepackage{bookmark}
\usepackage{enumitem}
\usepackage{float}


%%------------------Beginning of the text-----------------%%
\begin{document}

%%-----------------Cover------------------%%
\cover
\thispagestyle{empty}% Home page does not show page numbers


%%-----------------Catalog-------------------%%
\newpage
\tableofcontents

%%-----------------Main text starts here-------------------%%
\newpage
\section{习题}

\subsection{地址前缀匹配}
\textbf{问:}
以下地址前缀中的哪一个地址与$2.52.90.140$ 匹配?请说明理由。
\begin{enumerate}[label=\Roman*),leftmargin=2.2\parindent]
    \item 0/4; 
    \item 32/4; 
    \item 4/6; 
    \item 80/4。
\end{enumerate}

\textbf{答:}
$2.52.90.140$的二进制表示为:00000010.00110100.01011010.10001100
\begin{enumerate}
    \item 0/4 的二进制表示为:00000000.00000000.00000000.00000000
    \item 32/4 的二进制表示为:00100000.00000000.00000000.00000000
    \item 4/6 的二进制表示为:00000100.00000000.00000000.00000000
    \item 80/4 的二进制表示为:01010000.00000000.00000000.00000000
\end{enumerate}
因此,只有0/4的前缀与$2.52.90.140$的前缀的前四位相匹配


\subsection{掩码与网络前缀}
\textbf{问:}
与下列掩码相对应的网络前缀各有多少位?
\begin{enumerate}[label=\Roman*),leftmargin=2.2\parindent]
    \item $192.0.0.0$;
    \item $240.0.0.0$;
    \item $255.224.0.0$;
    \item $255.255.255.252$。 
\end{enumerate}

\textbf{答:}
子网掩码中1的个数即为网络前缀的位数。
\begin{enumerate}
    \item $192.0.0.0$ 的二进制表示为:11000000.00000000.00000000.00000000, 所以其网络前缀为2位
    \item $240.0.0$ 的二进制表示为:11110000.00000000.00000000.00000000, 所以其网络前缀为4位
    \item $255.224.0.0$ 的二进制表示为:11111111.11100000.00000000.00000000, 所以其网络前缀为11位
    \item $255.255.255.252$ 的二进制表示为:11111111.11111111.11111111.11111100, 所以其网络前缀为30位
\end{enumerate}

\subsection{路由表更新}
\textbf{问:}
假定网络中的路由器B的路由表如表\ref{tab:router_b}所示:
\begin{table}[h!]
    \centering
    \caption{路由器B的路由表}
    \begin{tabular}{c|c|c}
    \toprule
    目的网络 & 距离 & 下一跳路由器\\
    \midrule
    $N_1$ & $7$ & $A$\\
    $N_2$ & $2$ & $C$\\
    $N_6$ & $8$ & $F$\\
    $N_8$ & $4$ & $E$\\
    $N_9$ & $4$ & $F$\\
    \bottomrule
    \end{tabular}
    \label{tab:router_b}
\end{table}

现在B收到从C发来的路由信息,如表\ref{tab:updates}所示:
\begin{table}[h!]
    \centering
    \caption{路由器C至路由器B的更新信息}
    \begin{tabular}{c|c}
    \toprule
    目的网络 & 距离\\
    \midrule
    $N_2$ & $4$\\
    $N_3$ & $8$\\
    $N_6$ & $4$\\
    $N_8$ & $3$\\
    $N_9$ & $5$\\
    \bottomrule
    \end{tabular}
    \label{tab:updates}
\end{table}

试求出路由器B更新后的路由表(详细说明每一个步骤)。

\textbf{答:}
TODO

\subsection{IPv4地址转换}
\textbf{问:}
试把下列IPv4 地址从二进制记法转换为点分十进制记法:
\begin{enumerate}[label=\Roman*)]
    \item $10000001 \ 00001011 \ 00001011 \ 11101111$
    \item $11000001 \ 10000011 \ 00011011 \ 11111111$
    \item $11100111 \ 11011011 \ 10001011 \ 01101111$
    \item $11111001 \ 10011011 \ 11111011 \ 00001111$
\end{enumerate}

\textbf{答:}
TODO

\subsection{IPv4过渡至IPv6}
\textbf{问:}
从IPv4过渡到IPv6的方法有哪些?

\textbf{答:}
TODO

\end{document}
