% !TEX program = xelatex
\documentclass[12pt,hyperref,a4paper,UTF8]{ctexart}
\usepackage{CityUhomework}
\usepackage{booktabs}
\usepackage{bookmark}
\usepackage{enumitem}
\usepackage{float}


%%------------------Beginning of the text-----------------%%
\begin{document}

%%-----------------Cover------------------%%
\cover
\thispagestyle{empty}% Home page does not show page numbers


%%-----------------Catalog-------------------%%
\newpage
\tableofcontents

%%-----------------Main text starts here-------------------%%
\newpage
\section{习题}

\section{习题}

\subsection{UDP \& IP - 1}
\textbf{问:}
某个应用进程使用运输层的用户数据报UDP, 然后继续向下交给IP层后,又封
装成IP数据报。既然都是数据报,是否可以跳过UDP而直接交给IP层?哪些
功能UDP提供了但IP没有提供?

\textbf{答:}
TODO

\subsection{UDP \& IP - 2}
\textbf{问:}
一个UDP 用户数据报的数据字段为8192 字节。在链路层要使用以太网来传送。
试问应当划分为几个IP 数据报片?说明每一个IP 数据报片的数据字段长度和片
偏移字段的值。

\textbf{答:}
TODO

\subsection{停止等待协议}
\textbf{问:}
在停止等待协议中如果不使用编号是否可行?为什么?

\textbf{答:}
TODO

\subsection{连续ARQ协议}
\textbf{问:}
假定使用连续ARQ 协议,发送窗口大小是$3$, 而序号范围是$[0, 15]$,而传输媒体保证在接收方能够按序收到分组。在某一时刻,在接收方,下一个期望收到的
序号是$5$。试问:
\begin{enumerate}[label=\Roman*),leftmargin=2.2\parindent]
    \item 在发送方的发送窗口中可能出现的序号组合有哪些?
    \item 接收方已经发送出的、但仍滞留在网络中(即还未到达发送方)的确认分组可能有哪些?说明这些确认分组是用来确认哪些序号的分组。
\end{enumerate}

\textbf{答:}
TODO

\subsection{TCP报文}
\textbf{问:}
主机A向主机B连续发送了两个TCP报文段,其序号分别是70和100。试问:
\begin{enumerate}[label=\Roman*),leftmargin=2.2\parindent]
    \item 第一个报文段携带了多少字节的数据?
    \item 主机B收到第一个报文段后发回的确认中的确认号应当是多少?
    \item 如果B收到第二个报文段后发回的确认中的确认号是180,试问A发送的第二个报文段中的数据有多少字节?
    \item 如果A发送的第一个报文段丢失了,但第二个报文段到达了B。B在第二个报文段到达后向A发送确认。试问这个确认号应为多少?
\end{enumerate}

\textbf{答:}
TODO

\subsection{TCP信道}
\textbf{问:}
通信信道带宽为$1$Gbit/s, 端到端传播时延为10ms。TCP的发送窗口为$65535$字节。试问:
\begin{enumerate}[label=\Roman*),leftmargin=2.2\parindent]
    \item 可能达到的最大吞吐量是多少?
    \item 信道的利用率是多少?
\end{enumerate}

\textbf{答:}
TODO


\end{document}
