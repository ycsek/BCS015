% !TEX program = xelatex
\documentclass[12pt,hyperref,a4paper,UTF8]{ctexart}
\usepackage{CityUhomework}
\usepackage{booktabs}
\usepackage{bookmark}
\usepackage{enumitem}
\usepackage{float}


%%------------------Beginning of the text-----------------%%
\begin{document}

%%-----------------Cover------------------%%
\cover
\thispagestyle{empty}% Home page does not show page numbers


%%-----------------Catalog-------------------%%
\newpage
\tableofcontents

%%-----------------Main text starts here-------------------%%
\newpage

\section{习题}

\subsection{ARP-1}
\textbf{问:}
\begin{enumerate}[label=\Roman*,leftmargin=2.2\parindent]
    \item 试解释为什么ARP高速缓存每存入一个项目就要设置10$\sim$20分钟的超时计时器。这个时间设置得太大或太小会出现什么问题?

    \item 至少举出两种不需要发送ARP 请求分组的情况(即不需要请求将某个目的IP地址解析为相应的MAC地址)。
\end{enumerate}

\textbf{答:}
\begin{enumerate}[label=\Roman*,leftmargin=2.2\parindent]
    \item 根据ARP协议,我们可以将已知的IP地址转换为对应的MAC地址,当网络中的IP地址与MAC地址的映射关系发生变化时ARP cache中就会发生更改。若时间设置得太大,当网络中的IP地址与MAC地址的映射关系发生变化时ARP cache中的信息就会过时,导致无法通信。若时间设置得太小,ARP cache中的信息就会频繁更新,导致网络负载过大。

    \item 至少有两种不需要发送ARP 请求分组的情况:
    \begin{enumerate}[label=(\alph*)]
        \item 源主机的ARP cache中已经存有目标IP地址对应的MAC地址,那么就不需要再发送ARP请求。缓存中的信息可以直接用于数据包的发送。
        \item 源主机和目的主机使用点对点连接,不需要经过交换机或路由器,那么就不需要发送ARP请求。
        \item 源主机使用广播分组时,会发送至所有主机,不需要知道特定的MAC地址,所以不需要发送ARP请求。
    \end{enumerate}
\end{enumerate}

\subsection{ARP-2}
\textbf{问:}
主机A发送IP数据报给主机B,途中经过了5 个路由器。试问在IP数据报的发送过程中总共使用了几次ARP?

\textbf{答:}
6次,每经过一个路由器转发IP数据报就会使用一次ARP,主机发送数据报时会使用一次。


\newpage
\subsection{路由器转发表}
\textbf{问:}
设某路由器建立了转发表\ref{tab:转发表}:
\begin{table}[!ht]
    \centering
    \renewcommand{\arraystretch}{1}
    \setlength{\tabcolsep}{30pt}
    \begin{tabular}{l|r}
    \toprule
    前缀匹配    &   下一跳\\
    \midrule
    192.4.153.0/26  &   $R_3$\\
    128.96.39.0/25  &   接口$m_0$\\
    128.96.39.128/25    &   接口$m_1$\\
    128.96.40.0 /25 &   $R_2$\\
    192.4.153.0/26  &   $R_3$\\
    *(默认)   &   $R_4$\\
    \bottomrule
    \end{tabular}
    \caption{转发表}
    \label{tab:转发表}
\end{table}

现共收到5个分组,其目的地址分别为:
\begin{enumerate}[label=(\arabic*),leftmargin=2.2\parindent]
    \item 128.96.39.10
    \item 128.96.40.12
    \item 128.96.40.151
    \item 192.4.153.17
    \item 192.4.153.90
\end{enumerate}

试分别计算其下一跳。

\textbf{答:}
\begin{enumerate}[label=(\arabic*),leftmargin=2.2\parindent]
    \item 与转发表第二行匹配,下一跳为接口$m_0$
    \begin{table}[H]
        \centering
        \begin{tabular}{c|c}
            \toprule
            目的地址 & 128.96.39.10\\
            32位IP地址 & 10000000.01100000.00100111.00001010\\ 
            转发表第二行子网掩码 & 11111111.11111111.11111111.10000000\\
            与运算 & 10000000.01100000.00100111.00000000\\
            IP & 128.96.39.0\\
            \bottomrule
        \end{tabular}
        \caption{128.96.39.10}\label{128.96.39.10}
    \end{table}

    \newpage
    \item 与转发表第四行匹配,下一跳为$R_2$
    \begin{table}[H]
        \centering
        \begin{tabular}{c|c}
            \toprule
            目的地址 & 128.96.40.12\\
            32位IP地址 & 10000000.01100000.00101000.00001100\\
            转发表第四行子网掩码 & 11111111.11111111.11111111.10000000\\
            与运算 & 10000000.01100000.00101000.00000000\\
            IP & 128.96.40\\
            \bottomrule
        \end{tabular}
        \caption{128.96.40.12}\label{128.96.40.12}
    \end{table}

    \item 检查转发表第四行,结果不匹配,下一跳为默认接口$R_4$
    \begin{table}[H]
        \centering
        \begin{tabular}{c|c}
            \toprule
            目的地址 & 128.96.40.151\\
            32位IP地址 & 10000000.01100000.00101000.10010111\\
            转发表第四行子网掩码 & 11111111.11111111.11111111.10000000\\
            与运算 & 10000000.01100000.00101000.10000000\\
            IP & 128.96.40.128\\
            \bottomrule
        \end{tabular}
        \caption{128.96.10.151}\label{128.96.40.151}
    \end{table}

    \item 与转发表第一行匹配,下一跳为$R_3$
    \begin{table}[h]
        \centering
        \begin{tabular}{c|c}
            \toprule
            目的地址 & 192.4.153.17\\
            32位IP地址 & 11000000.00000100.10011001.00010001\\
            转发表第一行子网掩码 & 11111111.11111111.11111111.11000000\\
            与运算 & 11000000.00000100.10011001.00000000\\
            IP & 192.4.153.0\\
            \bottomrule
        \end{tabular}
        \caption{192.4.153.17}\label{192.4.153.17}
    \end{table}

    \item 检查转发表第一行,结果不匹配,下一跳为默认接口$R_4$
    \begin{table}[h]
        \centering
        \begin{tabular}{c|c}
            \toprule
            目的地址 & 192.4.153.90\\
            32位IP地址 & 11000000.00000100.10011001.01011010\\
            转发表第一行子网掩码 & 11111111.11111111.11111111.11000000 \\
            与运算 & 11000000.00000100.10011001.01000000\\
            IP & 192.4.153.64\\
            \bottomrule
        \end{tabular}
        \caption{192.4.153.90}\label{192.4.153.90}
    \end{table}

\end{enumerate}

\subsection{最大/最小IP}
\textbf{问:}
某单位分配到一个地址块$129.250/16$。该单位有$4000$台计算机,平均分布在$16$个不同的地点。试给每一个地点分配一个地址块,并算出每个地址块中IP地址的最小值和最大值。

\textbf{答:}
4000台计算机平均分布在16个地点,每个地点有$4000/16=250$台计算机。$2^8 = 256$所以主机号需要8位即可,网络前缀就为24位。16个地点,需要16个地址块,每个地址块有$2^8=256$个IP地址。所以每个地址块的地址范围如下:
\begin{table}[h]
    \centering
    \begin{tabular}{c|c}
        \toprule
        地址块 & IP地址范围\\
        \midrule
        129.250.1/24 & 129.250.1.0 - 129.205.1.255\\
        129.250.2/24 & 129.250.2.0 - 129.250.2.255\\
        129.250.3/24 & 129.250.3.0 - 129.250.3.255\\
        129.250.4/24 & 129.250.4.0 - 129.250.4.255\\
        129.250.5/24 & 129.250.5.0 - 129.250.5.255\\
        129.250.6/24 & 129.250.6.0 - 129.250.6.255\\
        \vdots & \vdots\\
        129.250.15/24 & 129.250.15.0 - 129.250.15.255\\
        129.250.16/24 & 129.250.16.0 - 129.250.16.255\\
        \bottomrule
    \end{tabular}
    \caption{IP地址范围}\label{IP地址范围}
\end{table}


\subsection{数据报}
\textbf{问:}
一个数据报长度为$4000$字节(固定首部长度)。现在经过一个网络传送,但此网络能够传送的最大数据长度为$1500$字节。试问应当划分为几个短些的数据报片?各数据报片的数据字段长度、片偏移字段和MF标志应为何数值?

\textbf{答:}
IP首部长度固定为20字节,则数据部分的长度为$4000-20=3980B$, 由于网络能够传送的最大数据长度为$1500$字节,所以每个数据报片的数据字段长度为$1500-20=1480B$。划分出一个数据报片为$3980-1480=2500B$,因其长度大于MTU所以还应再划分,即$2500-1480=1020B$。共划分3个数据报片,每个数据报片的数据字段的长度分别为:1480B, 1480B, 1020B。

片偏移字段分别为:$0$, $\frac{1480}{8}=185$, $\frac{2 \times 1480}{8} =370$。MF表示分片后是否还有数据报,1表示还有数据报,0表示已是最后一个数据报片。所以,MF标志分别为:1, 1, 0。

\end{document}
